%!TEX root = paper.tex

\section{Fault Tolerant Streaming Dataflows}
\label{sec:execution}
\para{The Job Graph.} Flink's execution model is based on the Job Graph, a directed acyclic graph (DAG) that consists of nodes and edges. There are two classes of nodes: (stateful) operators, and (logical) intermediate results (IRs) as presented on Figure \autoref{fig:JobGraph}. For example, the aforementioned graph consists of five operators (circles), and three intermediate results (cylinders). Operators abstract computation (e.g., transformations, joins, etc), state (e.g., a persistent counter), as well as data sources (e.g. reading data from a file system, a socket, a message queue, etc.), and data sinks. Operators produce intermediate results, as well as updates to state. An intermediate result is a logical handle (pointer) to the data that is produced by one operator. An intermediate result can be consumed by one or more operators. Intermediate results are logical in the sense that the data they point to may or may not be materialized on disk. When the JobGraph is scheduled in a cluster for execution, it is parallelized to form an ExecutionGraph that consists of tasks (parallel instances of operators), and intermediate result partitions (IRPs).


\para{Data Transfer.} The unit of data transfer in the Flink runtime is a buffer. Buffers contain one or more records, and a record can span multiple buffers. Buffers are requested and relinquished from local buffer pools, shared among operators that live in the same task manager. An IRP is simply a collection of buffers. Work in Flink progresses (i.e., records flow through the pipeline) as long as there are buffers available, essentially implementing distributed blocking queues (the logical streams) with bounded capacity (the amount of memory available to the buffer pools, which can be configured by the user). This mechanism, in addition to implementing record network transfers doubles down as a natural way to backpressure the flow in the case of slow operators (including external systems that consume data). 

We mentioned that the intermediate results are logical handles to the data, rather than the data itself. Internally, intermediate results are abstract classes with many implementations. These implementations can perform pipelined data exchange, and blocking data exchange.

\para{Pipelined Data Exchange.} Pipelining (also called intra-operator parallelism), means that a producing and a consuming operator make progress at the same time, without the consumer waiting for the producer to finish. Pipelining is required in streaming systems, and is also used in batch systems to reduce latency. Flink implements pipelining by implementing intermediate results which activate network transfers between the producer task and consumer task,  as soon as their first buffer is available. Flink allows the configuration of its buffers' size and timeout. A buffer is sent to its consumer task as soon it is filled,  or as soon as a timeout is reached. Hence, by setting the buffer size and/or timeout to lower values, one can reduce latency, while achieving the opposite effect (i.e., high throughput) by using higher values.

\para{Blocking Data Exchange.} Sometimes it is desirable to break a job into stages, scheduling and executing each stage individually (e.g., to enable interactive processing, and better staging of resources in large batch jobs). To do that, the system has to materialize intermediate results (in memory or disk). Flink implements blocking data exchange via an intermediate result that signals its availability only when all the buffers from the producer have been materialized. The cached buffers can double down as a materialized reusable intermediate result. 

